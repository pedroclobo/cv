\documentclass[a4paper,11pt]{article}

\usepackage{geometry}
\geometry{
  inner=1in,
  outer=1in,
}
\usepackage[english]{babel}
\usepackage{fontawesome5}
\usepackage{titlesec}
\usepackage{xcolor}
\usepackage{enumitem}
\usepackage{hyperref}
\hypersetup{
  colorlinks=true,
  urlcolor=[HTML]{0000EE},
}

\usepackage{fontspec}
\usepackage{libertine}
\setmonofont{Inconsolatazi4}

\usepackage{changepage}

\setlist[itemize]{
  rightmargin=5pt,
}

% Remove page numbers
\pagestyle{empty}

% Better formatting for sections
\titleformat{\section}{
  \vspace{-4pt}\scshape\raggedright\Large
}{}{0em}{}[\color{black}\titlerule \vspace{-5pt}]

% Replace itemize bullets with circles
\renewcommand{\labelitemii}{$\circ$}

\begin{document}

% Heading
\noindent
\begin{minipage}{\textwidth}
  \begin{adjustwidth}{-6pt}{0pt}
    \begin{tabular*}{\dimexpr\textwidth+10pt\relax}{l@{\extracolsep{\fill}}c@{\extracolsep{\fill}}r}
      \textbf{\raggedright\LARGE Pedro Lobo} &
      \href{https://github.com/pedroclobo}{\faGithub} \hspace{6pt}
      \href{https://www.linkedin.com/in/pedroclobo/}{\faLinkedin} \hspace{6pt}
      \href{mailto:pedrocerqueiralobo@gmail.com}{\faEnvelope} &
      \textbf{\Large Software Engineer}\\\\
    \end{tabular*}
  \end{adjustwidth}
\end{minipage}

\section{Education}
  \begin{itemize}[leftmargin=*]
    \vspace{-1pt}
    \item
    \begin{tabular*}{0.97\textwidth}[t]{l@{\extracolsep{\fill}}r}
      \textbf{Instituto Superior Técnico} & Lisbon, Portugal \\
      \textit{\small Master in Computer Science and Engineering} & \textit{\small September 2023 -- November 2025} \\
      \textit{\small Undergraduate in Computer Science and Engineering} & \textit{\small September 2020 -- July 2023} \\
    \end{tabular*}\vspace{-5pt}
      \begin{itemize}
        \item\small{Academic Excellence Diploma (2022/2023)}
        \item\small{Academic Merit Diploma      (2021/2022)}
        \item\small{Academic Merit Diploma      (2020/2021)}
      \end{itemize}
  \end{itemize}

\section{Experience}
  \begin{itemize}[leftmargin=*]
    \vspace{-1pt}
    \item
    \begin{tabular*}{0.97\textwidth}[t]{l@{\extracolsep{\fill}}r}
      \textbf{Instituto Superior Técnico} & Lisbon, Portugal \\
      \textit{\small Invited Teaching Assistant} & \textit{\small September 2023 - July 2024} \\
    \end{tabular*}\vspace{-5pt}
      \begin{itemize}
        \item\small{
          \textbf{Compilers:} {
            Due to my strong performance in this course, I was invited to teach practical classes for the following year's students.
            I covered topics ranging from lexical analysis to code generation and optimization.
            Additionally, I held office hours to provide extra support and assisted students with their final projects.
          \vspace{-2pt}}
        }
        \item\small{
          \textbf{Foundations of Programming:}{
            I was responsible for teaching practical classes, grading students' weekly assignments, providing project support, offering additional guidance through office hours, and evaluating the two projects developed by students throughout the semester.
          \vspace{-2pt}}
        }
      \end{itemize}
  \end{itemize}

\section{Open Source Contributions}
 \begin{itemize}[leftmargin=*]
  \item\small{
    \textbf{\href{https://github.com/llvm/llvm-project/pulls?q=author:pedroclobo}{LLVM Project:}}{
      The LLVM Project is a widely used collection of modular and reusable compiler and toolchain technologies, powering the code generation of many programming languages such as C, C++, Rust and Swift.
      I regularly contribute to LLVM, mainly helping with the \texttt{undef} to \texttt{poison} transition, \href{https://github.com/llvm/llvm-project/pulls?q=author:pedroclobo}{having merged around 30 PRs}.
      These contributions have granted me \href{https://github.com/llvm/llvm-project/issues/129447}{commit access to the LLVM project}.
    \vspace{-2pt}}
  }\vspace{-4pt}
  \item\small{
    \textbf{\href{https://github.com/leic-pt/resumos-leic/pulls?q=author:pedroclobo}{Resumos LEIC:}}{
      An open source website, dedicated to the creation of class notes focused on the courses in the curriculum of the undergraduate in Computer Science and Engineering at Técnico Lisboa, improving the academic performance of students.
      I contributed class notes for the courses of Machine Learning, Systems Analysis and Modeling, Computer Organization and Elements of Discrete Mathematics.
    \vspace{-2pt}}
  }\vspace{-4pt}
 \end{itemize}

\section{Projects}
  \begin{itemize}[leftmargin=*]
    \item\small{
      \textbf{\href{https://github.com/pedroclobo/monkey-interpreter}{Monkey Interpreter:}}{
        Interpreter and REPL for a toy language, written in Rust, with a hand written lexer and parser for C-like syntax, featuring strings, arrays, hashmaps, first-class functions and higher-order predicates.
        I use this project as a playground to test and implement new ideas and concepts.
      \vspace{-2pt}}
    }\vspace{-4pt}
    \item\small{
      \textbf{\href{https://github.com/pedroclobo/cs-6120}{CS 6120:}}{
        Playground for the \href{https://www.cs.cornell.edu/courses/cs6120/2020fa/self-guided/}{CS 6120 Course on Advanced Compilers}.
        I have implemented various optimizations for a simple intermediate representation called Bril, such as dead code elimination, constant folding and local value numbering.
      \vspace{-2pt}}
    }\vspace{-4pt}
  \end{itemize}

\end{document}
